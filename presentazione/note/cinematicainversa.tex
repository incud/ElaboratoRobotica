\note[itemize]{
    \item proponiamo un insieme di algoritmi per il calcolo della cinematica inversa
    \item dobbiamo essere in grado di valutare ognuno di questi e confrontarlo per decidere il migliore
    \item ci focalizziamo su due aspetti
    \item efficienza nelle tempistiche
    \item efficacie nelle trovare la configurazione dei giunti per il maggior numeri di punti raggiungibili
    \item riguardo quest'ultimo punto, chiaramente non possiamo testare l'algoritmo per ogni singolo punto dello spazio
    \item per prima cosa calcoliamo lo spazio raggiungibile dal manipolatore, che è una semisfera di raggio $r$ ottenuto come somma delle lunghezze dei bracci (nell'immagine è presente una fetta di questo spazio)
    \item in un primo momento, prendiamo un insieme di punti all'interno $N$ punti in rosso da raggiungere con in manipolatore
    \item in un secondo momento vogliamo invece andare oltre e indicare al manipolatore anche la forma desiderata dell'intero robot. facciamo questo specificando dei punti intermedi in blu - per esempio possiamo trovarci a dover raggiungere un punto rosso passando il più possibile vicino ad un punto blu
    \item questo però in un secondo momento
}

\note[itemize]{
    \item modelliamo il problema di minimizzazione
    \item la funzione di costo ha in input una configurazione di giunti, ed in output da la distanza tra il punto desiderato e il punto raggiunto dalla configurazione in input
    \item confrontato con TRAC-IK, libreria che vanta essere la migliore implementazione di cinematica inversa
}
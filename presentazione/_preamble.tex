\usepackage[italian]{babel}
\usepackage{pgfpages}
\usepackage{verbatim}
\setbeamertemplate{note page}[plain]

% Tikz
\usepackage{tikz}
\usepackage{pgfplots}
\usetikzlibrary{positioning,snakes,calc,arrows,decorations.markings,shapes.misc,matrix,shapes,arrows,fit,tikzmark,patterns}

% Fonts
\usepackage{helvet}
\usepackage[default]{lato}
\setbeamertemplate{caption}{\raggedright\insertcaption\par}

% Grafica
\usepackage{graphicx}
\usepackage[space]{grffile}
\usepackage[clock]{ifsym}

\usepackage{savesym}
\savesymbol{checkmark}
\usepackage{dingbat}

% Colori
\definecolor{blue}{RGB}{0,114,178}
\definecolor{lightblue}{RGB}{80,194,255}
\definecolor{red}{RGB}{213,94,0}
\definecolor{yellow}{RGB}{240,228,66}
\definecolor{green}{RGB}{0,158,115}
\definecolor{black}{RGB}{0,0,0}

\setbeamercolor{normal text}{fg=white,bg=black}
\setbeamercolor{frametitle}{fg=lightblue}
\setbeamercolor{title}{fg=white}
\setbeamertemplate{footline}[frame number]
\setbeamertemplate{navigation symbols}{} 
\setbeamertemplate{itemize items}{-}
\setbeamercolor{itemize item}{fg=yellow}
\setbeamercolor{itemize subitem}{fg=yellow}
\setbeamercolor{enumerate item}{fg=yellow}
\setbeamercolor{enumerate subitem}{fg=yellow}
%\setbeamercolor{button}{bg=MyBackground,fg=blueblue,}
\setbeamercolor{section in toc}{fg=yellow}
\setbeamercolor{subsection in toc}{fg=red}
\setbeamersize{text margin left=3em,text margin right=3em}

% Links
\usepackage{hyperref}
\hypersetup{
  colorlinks=false,
  linkbordercolor = {white},
  linkcolor = {blue}
}

% Math
\usepackage{mathpazo}
\usepackage{amsmath,amssymb,amsfonts,mathtools,IEEEtrantools,siunitx,cancel}

\DeclareMathOperator*{\argmax}{arg\,max}
\DeclareMathOperator*{\argmin}{arg\,min}

% Transition Frame
\newenvironment{transitionframeenv}{%
\setbeamercolor{%
background canvas}{bg=blue}%
\begin{frame}}%
{\end{frame}}%

\newcommand{\TransitionFrame}[1]{%
\begin{transitionframeenv}%
\begin{center}%
{\Huge\textcolor{white}{#1}}%
\end{center}%
\end{transitionframeenv}}

\newcommand{\Definizione}[2]{%
\textcolor{yellow}{{#1}}

\medskip\noindent{#2}%
}

\usepackage[absolute,overlay]{textpos}

\newcommand{\btVFill}{\vskip0pt plus 1filll}

\newcommand{\Reference}[3]{\begin{textblock*}{\textwidth}(1cm,8cm)%
\footnotesize Reference: \emph{#1}, #2, #3
\end{textblock*}}

\newcommand{\BReference}[3]{\begin{textblock*}{\textwidth}(1cm,8.5cm)%
\footnotesize Reference: \emph{#1}, #2, #3
\end{textblock*}}

\usepackage{url}